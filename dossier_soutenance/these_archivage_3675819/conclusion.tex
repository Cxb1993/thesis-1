\setchapterpreamble[u]{\margintoc}
\chapter{Conclusions \& Perspectives}
\labch{conclusion}

In this thesis, numerical investigations pertaining to two different aspects 
of liquid-gas interfacial flows with marked density contrasts were conducted 
using advanced geometrical reconstruction based Volume-of-Fluid methods. 
The first aspect involved understanding and eventually suppressing certain types of 
instabilities that plague standard numerical models due to the cascading growth of
discretization errors that are particularly rampant at low to moderate numerical resolution.  
On the other hand, the second aspect of the investigation deals with a statistical description
of drop sizes, where the drops were generated by carrying out large ensembles of well resolved 
direct numerical simulations of the capillary-induced breakup of randomly corrugated ligaments. 

In the first part of the thesis, a detailed exposition into the design and 
development of our numerical methods is provided, where the principle 
of maintaining consistency between the discrete transport of
the discontinuous mass and momentum fields was implemented  
using a conservative discretization of the two phase 
Navier-Stokes equations on staggered Cartesian grids. 
The \textit{shifted fractions} and \textit{sub-grid} methods were developed
specifically in order to address the challenge of consistent 
advection of volume fraction, mass and momentum on staggered grids,
coupled with the extension of the conservative direction-split mass transport
algorithm of Weymouth and Yue \cite{wy} to the direction-split transport of momentum.
Several performance aspects of the aforementioned methods were presented using 
standard benchmark cases such as the behavior of spurious currents in static and
moving droplets, and the propagation of damped capillary waves. 
Each of these tests were conducted for density contrasts spanning several orders of magnitude, 
including a realistic air-water configuration in the case of capillary waves. 
Both of our consistent methods demonstrate moderate to significant levels of improvement 
in accuracy when compared to the standard version of our method that is not 
consistent with respect to mass-momentum transport. 
A second-order error convergence rate is observed in the case of the 
static droplet and the capillary wave, with the rate
reverting to first-order in the case of droplet advection.
Finally, the robustness and stability of the present method were demonstrated
using the case of a raindrop falling in air, which involves complex interactions
of capillary forces, viscosity and inertia combined with the difficulties associated
with flows involving marked density contrasts.  
A key feature of our class of consistent methods is the ability to circumvent 
or suppress the un-physical interfacial deformations that are rampant 
in the case of the standard non-consistent method, which typically results in the 
``artificial'' atomization of the raindrop .
Our methods are able to deliver relatively good estimates of the flow features 
especially at low to moderate resolutions, as evidenced by the smooth evolution 
of the droplet kinetic energy and the moments of inertia describing the average shape. 
Furthermore, both methods are able to reproduce the physically consistent early phase 
(almost constant) acceleration of the droplet towards the final statistical steady state.  

Many of the complications in our methods are due to the staggered arrangement of the primary variables, 
thus opening a possible avenue for exploration of different grid configurations when it comes to consistent transport.
Another perspective is an assessment of the computational speed-up 
of our methods compared to standard non-consistent methods which necessitate 
significantly higher spatial resolutions in order to ensure numerical stability.  
An interesting possibility exists in the context of the \textit{sub-grid} method, 
where volume fraction information from the doubly refined grid could be used to
obtain more accurate estimations of the interface curvature defined on the coarse grid level. 
An interesting test would be to verify if both the consistent and non-consistent methods
converge to identical solutions at sufficiently large mesh resolution, owing to the fact that they 
are based on identical governing equations and differ mainly due to the choice of discretization scheme. 
In view of the pervasiveness of shear-driven primary instabilities in a wide range 
of fragmentation phenomena, a long term perspective would be to thoroughly evaluate 
and subsequently improve the accuracy of our consistent methods in comparison to the theoretical 
growth rate predictions (linear theory) of the Kelvin-Helmholtz instability for flows
involving marked density contrasts, in a manner akin to our recent study \cite{caf_momcons}. 

Moving onto the second part of the thesis, a thorough description of the 
methodology behind the generation of slender periodic ligaments is provided, 
where the ligament surfaces are created via superposition of random overlapping waves.
A numerical setup was developed that enables precise quantitative control over the 
statistical properties of the randomly corrugated initial surface profiles, 
therefore setting up the stage for obtaining controllable, reliable and reproducible 
ensembles of droplets that are generated via disintegration of the original ligaments.
High fidelity simulations of ligaments corresponding to air-water systems were carried out,
delineating the influence of important factors such as the length scale, time scale, 
aspect ratio and strength of the initial corrugations on the subsequent set of drops formed.  
In the case of moderately perturbed millimeter scale water ligaments in air, 
the diameters of the resulting drops were found to be (approximately) symmetrically 
distributed around the mean size, where the overall shape is relatively well described by 
a Gaussian probability density function devoid of any free parameters.
The shape of the diameter distributions is fairly robust, displaying negligible sensitivity
to the choice of bin-width used to generate the histograms.
In addition, the distributions converge towards a fixed shape as we increase the sample size $N$, 
where the error (defined with respect to the largest sample) scales as $N^{-1/2}$, therefore strongly 
suggesting the absence of any hidden correlations between the individual realizations constituting the sample. 
Amongst the initial conditions that influence the shape of the diameter distribution, the strength of 
the corrugations acts as a bifurcation parameter, changing the bimodal nature of the distribution shape for 
weakly perturbed ligaments to a unimodal one as we go towards stronger initial perturbations. 
Interestingly, in the case of moderately perturbed ligaments, the tail of the diameter 
distribution corresponding to the large drops is best described by an exponential scaling of
the form $e^{-d^6}$ even though the overall distribution seems to scale as $e^{-d^2}$. 
This scaling result was found to be independent of whether uncertainty in the bin heights
was included or not, and further verified by reconstructing the histograms on "$\log(\log)$ vs. $\log$" scales.      
An interesting perspective is to compare the quality of the fit associated with 
the exponential scaling to alternate forms such as log-normal, power law etc.  
Finally, a Gaussian random process theory for near-monochromatic waves was developed 
in an effort to explain the origins of the newfound $e^{-d^6}$ scaling.
In the surface tension dominated regime associated to weakly perturbed initial ligament shapes, 
the theory predicts a resulting drop diameter distribution in the form of a $d^3$- Gaussian, 
as well as a functional relationship between the dispersion in the size of the drops and the 
variance of the initial perturbations used to create the corrugated ligament surface. 
The theoretical predictions significantly differ from the observations derived from our numerical experiments
corresponding to \textit{moderately} perturbed ligament shapes, therefore opening the perspective
of verifying the predictions within its scope of validity i.e. adequately large ensembles of \textit{weakly} perturbed ligaments.

As we have devised a numerical setup that allows us to efficiently generate
large numbers of drops using high-fidelity simulations, an avenue for further investigation
lies in the exploration of ligament ensembles corresponding to 
larger aspect ratios, centimeter length scales and a broader range of initial corrugation strength. 
Such an undertaking may allow us to quantify the variation in the mean drop sizes as a function of the 
point in the parameter space $\Phi$, opening up the possibility that the combination of size distributions
characterized by different points in $\Phi$ could lead to more complicated and asymmetric probability
distributions, thereby providing an alternate interpretation to the skewness observed in the drop size 
distributions typical of more complex fragmentation processes. 
On a final note, towards the objective of rendering our ligament model more representative of  
realistic fragmentation scenarios, a long term perspective would be to include the added complexity
of inertial stretching as part of our initial conditions ($\textrm{We} > 1$).  
