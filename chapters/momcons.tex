\setchapterpreamble[u]{\margintoc}
\chapter{Consistent Mass-Momentum Transport}
\labch{momcons}

% intro : PARIS Simulator, necessity due to large density-ratios ? 

The stability of numerical methods that attempt to model 
interfacial flows involving large density contrasts 
face several challenges, key amongst them being the transport of 
mathematical discontinuities that arise from the contrast in
the material properties of the fluids involved. 
Extremely small numerical errors are ubiquitous as a consequence of the 
numerous approximations involved at each and every step of the algorithm 
\sidenote{Approximations are made while interfacial propagation
, curvature computation, surface tension modeling etc.}. 
In the context of such flows, such errors generally result in physically
inconsistent mass and momentum transfer across the interface, often from
the denser phase towards the lighter phase.
The presence of large density or viscosity contrasts tend to amplify
the growth of these cascading numerical errors, eventually
leading to significant (often catastrophic) interfacial deformations
followed rapidly by a loss of numerical stability. 
In this chapter, we present a detailed exposition of our class 
of numerical methods which are designed specifically to circumvent or
suppress the growth of such numerical instabilities when dealing with
flows entailing large density-ratios. 
The implementations of the numerical algorithms are developed for 
finite volume discretizations using uniform Cartesian grids in 3D, 
on the `PARIS Simulator' \cite{paris} numerical platform.      
Some contents of the chapter are primarily derived 
from the recent study by Fuster et al. \sidecite{caf_momcons}, 
in which the author is a primary contributor. 

% ----------------------------------------------------------------------------------------------------------------------

\section{General Principles}

%intro : what do you mean by consistent transport? 
In the past two decades, considerable efforts have been made 
in the design of numerical methods to deal with large density-ratios. 
The underlying principle behind these endeavours is that the governing 
equations for the transport of mass and momentum are modeled using a 
conservative formulation (divergence of fluxes), instead of standard 
non-conservative forms which themselves were adapted directly from 
techniques developed originally for single-phase flows. 
This formulation enables one to render the discrete transport of 
momentum \textit{consistent} with respect to the discrete transport of mass  
\sidenote{First of all, the transport of mass (density) has to be made
consistent with that of the geometric transport of volume fraction.}. 
Such a tight coupling of the propagation of errors between the discrete 
mass and momentum fields enables alleviation of many of the issues 
that plague such numerical methods, especially in the 
context of low to moderate spatial resolutions. 
A secondary but important mitigating factor is the advancements 
made in the modeling of capillary forces, resulting in the adoption of 
consistent and \textit{well-balanced} surface tension formulations.
\sidenote{Consistency in the context of surface tension models refers to the ability 
of methods to progressively achieve more accurate estimations of interfacial 
curvature as a result of increasing spatial resolution, 
whereas well-balanced refers to the ability to recover certain static 
equilibrium solutions pertaining to surface tension dominated flows 
without the perpetual presence of parasitic or spurious currents in the velocity fields.}.
We refer the reader to influential works of Popinet \cite{popinet2018numerical,popinet2009accurate}
to get a better understanding of the issues surrounding surface tension implementations.  

\subsection*{Major Iterations in Literature}

% literature review 

% Rudman (1998) :
The first study to address the issue of consistency between mass 
and momentum transport was the seminal work of Rudman \sidecite{rudman1998volume}.
The fundamental hurdle in the implementation of mass-momentum 
consistent transport for staggered configurations of primary variables 
(pressure and velocity) is the inherent difficulty in reconstructing mass 
(defined on centered control volumes) and its corresponding fluxes 
onto the staggered control volumes on which momentum is defined. 
Rudman introduced the strategy of carrying out mass advection 
\sidenote{Mass advection was carried out using using algebraic flux reconstructions.} 
on a grid twice as fine as that of momentum, 
thereby enabling a `natural' and intutive 
way to reconstruct mass and its fluxes onto staggered momentum control volumes. 
However, the method uses a VOF based convolution technique for curvature computation, 
which is neither consistent nor well-balanced.    


% Bussmann et al. (2002) :
Bussmann et al. \sidecite{bussmann2002modeling} were able to 
circumvent the issue surrounding staggered grids altogether 
by using a collocated arrangement in the context of hexahedral 
unstructured meshes, coupled with an unsplit Eulerian flux computation method.    
The study though makes no mention of any surface tension model. 


% Raessi & Pitsch (2012) | Ghods & Hermann (2013):
Level set based methods in the context of mass-momentum consistent 
transport were implemented first by Raessi and Pitsch \sidecite{raessi2012consistent}, 
followed by Ghods and Hermann \cite{ghods2013consistent}. 
In the former, the consistency problem is tackled by means of a 
semi-Lagrangian approach, computing geometric level set derived 
fluxes at two different time intervals, whereas in the latter, 
a collocated arrangement is used. 
Nonetheless, both methods face certain drawbacks, notably the 
applicability only to 2D in case of Raessi and Pitsch, 
as well as a lack of well-balanced surface tension models for both these methods.   


% LeChenadec & Pitsch (2013) | Owkes & Desjardins (2017) : 
Recent advances concerning volume-of-fluid based methods that 
employ unsplit (conservative) geometric flux reconstructions 
were made by LeChenadec and Pitsch \sidecite{le2013monotonicity}
, and later by Owkes and Desjardins \cite{owkes2017mass}. 
LeChenadec and Pitsch utilize a Lagrangian remap method in order 
to construct consistent mass-momentum fluxes for the staggered control volumes, 
while Owkes and Desjardins use mass advection on a doubly refined grid 
(same principle as Rudman) to achieve consistency.     
Although \cite{le2013monotonicity} implements a well-balanced surface tension model, 
the VOF convolution based curvature computation is not consistent. 
In case of \sidecite{owkes2017mass}, they use mesh-decoupled height functions 
to compute curvature while coupling it with a well-balanced surface tension model. 
However, their semi-Lagrangian flux computation procedure 
involving streak tubes and flux polyhedra are extremely convoluted in 3D.     


% Vaudor et al. (2017) | Zuzio et al. (2020) : 
Certain methods attempt to combine the qualities of both 
VOF and level set methodologies (CLSVOF), as proposed in the works of 
Vaudor et al. \sidecite{vaudor2017consistent}, 
and more recently by Zuzio et al. \cite{zuzio2020new}. 
They both tackle the consistency issue by means of projecting 
the direction-split geometric fluxes onto a twice finer grid, 
which are subsequently recombined to reconstruct consistent 
fluxes for mass and momentum for the staggered control volumes. 
This approach allows them to bypass the requirement of conducting 
mass advection on a twice finer grid (as in the original Rudman method), 
thereby deriving the benefits of a sub grid without the 
added computational cost of doubly refined mass transport. 
In addition, both Vaudor et al. \cite{vaudor2017consistent} and 
Zuzio et al. \sidecite{zuzio2020new} adopt well-balanced surface tension 
models with consistent level set based curvature estimation. 
However, the purported advantages of both these methods with 
regards to reduced computational costs is not quite evident, 
as additional complexities are introduced due to the projection 
(reconstruction) of fluxes onto a the twice finer mesh, 
which would not be necessary in the first place if mass transport 
had been carried out on the twice finer mesh itself. 


% Patel & Natarajan (2017) : 
Patel and Natarajan \sidecite{patel2017novel} developed a hybrid 
staggered-collocated approach to solve the consistency issue on 
polygonal unstructured meshes, complemented with a well-balanced surface tension model. 
Nevertheless, the VOF advection is based on algebraic transport, 
not to mention the use of a VOF convolution based 
curvature computation, which is inherently not consistent. 


% Nangia et el. (2019) : 
More recently, Nangia et al. \sidecite{nangia2019robust} developed a 
CSLVOF method for dynamically refined staggered Cartesian grids. 
They utilize Cubic Upwind Interpolation (CUI) schemes to 
reconstruct consistent mass and momentum fluxes on the staggered 
control volumes, using the information from the additional mass 
advection equation they solve alongside the level set function.   
However, the reconstruction of mass fluxes using CUI schemes are 
inherently algebraic, with their comparative advantage against fluxes 
computed via geometric constructions being an open question 
\sidenote{refer to Mirjalili et al. \cite{mirjalili2017interface}. }.

To get a bird's eye view of the numerous features employed by the methods 
in existing literature, we refer the reader to tables %\ref{table_vof} and \ref{table_ls}, 
which respectively provide systematic overviews of the VOF based and level set based approaches.    

% modified comparison charts with CAF paper added



\subsection*{Our Approach}

In the present body of work, we start by precisely defining the
essential (desired) attributes of a numerical scheme that 
ensures discrete consistency between mass and momentum transport : 

\begin{itemize}
	\item The discontinuity of the mass (volume fraction 
		derived density field) should propagate at the 
		exact `numerical' speed as that of 
		the discontinuity of the momentum field. 
	\item The numerical transport of momentum should be performed 
		in a manner consistent to the transport of mass for 
		each direction, implying that the momentum fluxes must 
		be obtained directly from geometrically 
		computed fluxes of mass (volume). 
\end{itemize}

In order to tackle the challenge of consistent transport on staggered control 
volume configurations, we have developed two different strategies, namely, 
the \textit{shifted fractions} method and the \textit{sub-grid} method. 
The former uses geometrical reconstructions to derive a 
\textit{shifted} volume fraction field which is centered on the staggered
control volumes, whereas the latter adopts the Rudman \cite{rudman1998volume} 
strategy of volume advection on a twice finer grid in order to 
enable consistent reconstruction of mass and momentum on the staggered control volumes. 
Another key contribution of this body of work is to extend the conservative 
direction-split mass transport algorithm of Weymouth and Yue \sidecite{wy} 
to the direction-split transport of momentum. 
In the ensuing chapter, we demonstrate that adoption of these strategies 
lead to significant improvements in terms of numerical stability and robustness,
therefore enabling the modeling of flows with more realistic density and 
viscosity ratios compared to the standard non-consistent formulations. 

% ----------------------------------------------------------------------------------------------------------------------


\section{Basic Expressions}

% intro : advection of a generic conserved quantity 

\subsection*{Advection of Density}

% VOF - rho consistency with compression term etc

\subsection*{Advection of Momentum}

% model of computing consistent mass-momentum fluxes

\subsection*{Advected Velocity}

% computation of slope limiters with diagram, distinction in bulk and interfacial regions


% ----------------------------------------------------------------------------------------------------------------------


\section{Staggered Reconstructions}

% intro : staggered configuration of primary variables, control volumes for staggered advection 

\subsection*{Shifted Fractions Method}

% algorithm, flow chart , how to reconstruct on shifted grid, lack of conservation 

\subsection*{Sub-Grid Method}

% algorithm, flow chart, prolongation and restriction operations, consistent + conservative 

\subsection*{Source Terms}

% 1/rho terms in ST , reconstructing curvatures from sub-grid information etc 
