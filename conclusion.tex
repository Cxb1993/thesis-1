\setchapterpreamble[u]{\margintoc}
\chapter{Conclusions \& Perspectives}
\labch{conclusion}

In this thesis, numerical investigations pertaining to two different aspects 
of liquid-gas interfacial flows with marked density contrasts were conducted 
using advanced geometrical reconstruction based Volume-of-Fluid methods. 
The first aspect involved understanding and eventually suppressing certain types of 
instabilities that plague standard numerical models due to the cascading growth of
discretization errors that are particularly rampant at low to moderate numerical resolution.  
On the other hand, the second aspect of the investigation deals with a statistical description
of drop sizes, where the drops were generated by carrying out large ensembles of well resolved 
direct numerical simulations of the capillary-induced breakup of randomly corrugated ligaments. 

In the first part of the thesis, a detailed exposition into the design and 
development of our numerical methods is provided, where the principle 
of maintaining consistency between the discrete transport of
the discontinuous mass and momentum fields were implemented  
using a conservative discretization of the two phase 
Navier-Stokes equations on staggered Cartesian grids. 
The \textit{shifted fractions} and \textit{sub-grid} methods were developed
specifically in order to address the challenge of consistent 
advection of volume fraction, mass and momentum on staggered grids,
coupled with the extension of the conservative direction-split mass transport
algorithm of Weymouth and Yue \cite{wy} to the direction-split transport of momentum.
Several performance aspects of the aforementioned methods were presented using 
standard benchmark cases such as the behavior of spurious currents in static and
moving droplets, and the propagation of damped capillary waves. 
Each of these tests were conducted for density contrasts spanning several orders of magnitude, 
including a realistic air-water configuration for the case of capillary waves. 
Both of our consistent methods demonstrate slight to moderate levels of increased accuracy
when compared to the standard version of our method that is not 
consistent with respect to mass-momentum transport. 
A second-order error convergence rate is observed in the case of the 
static droplet and the capillary wave, with the rate
reverting to first-order in the case of droplet advection.
Finally, the robustness and stability of the present method was demonstrated
using the case of a raindrop falling in air, which involves complex interactions
of capillary forces, viscosity and inertia combined with the difficulties associated
with flows involving marked density contrasts.  
A key feature of the our class of consistent methods is the ability to circumvent 
or suppress the un-physical interfacial deformations that are rampant 
in the case of the standard non-consistent method, which typically results in the 
``artificial'' atomization of the raindrop .
Our methods are able to deliver relatively good estimates of the flow features 
especially at low to moderate resolutions, as evidenced by the smooth evolution 
of the droplet kinetic energy and the moments of inertia describing the average shape. 
Furthermore, both methods are able to reproduce the physically consistent early phase 
(almost constant) acceleration of the droplet towards the final statistical steady state.  

Many of the complications in our methods are due to the staggered arrangement of the primary variables, 
thus opening a possible avenue for exploration of different grid configurations when it comes to consistent transport.
Another perspective is an assessment of the computational speed-up 
of our methods compared to standard non-consistent methods which necessitate 
significantly higher spatial resolutions in order to ensure numerical stability.  
An interesting possibility exists in the context of the \textit{sub-grid} method, 
where volume fraction information from the doubly refined grid could be used to
obtain more accurate estimations of the interface curvature defined on the coarse grid level. 
Finally, an interesting test would be to verify if both the consistent and non-consistent methods
converge to identical solutions at sufficiently large mesh resolution, owing to the fact that they 
are based on identical governing equations and differ mainly due to the choice of discretization scheme. 



